\begin{alphaparts}
    %----------------------------------------------------------------------------
   \questionpart
    Definiamo tutte le possibili partizioni \(o - d\) del grafo:

        \[ U_1 = \{o\}\]
        \[U_2 = \{o, a\}\]
        \[U_3 = \{o, b\}\]
        \[U_4 = \{o, a, b\}\]

    e tutti i possibili path \(o- d\):

        \[p_1 = \{o, a, d\}\]
        \[p_2 = \{o, b, d\}\]
        \[p_1 = \{o, a, b, d\}\]
    Affinché il flusso \(o- d\) si annulli deve essere rimossa almeno la seguente quantità di capacità:
    \[C_{\min}= \min((C_1 + \min(C_2, C_5)), (C_2+\min(C_1, C_3+ C_4))) \]

    Dove \(C_1 + \min(C_2, C_5)\) è la minima capacità da rimuovere nel caso in cui \(C_1> C_2\) mentre \(C_2+\min(C_1, C_3+ C_4)\) è la minima capacità da rimuovere nel caso in cui \(C_2 > C_1\)
    
    % --------------------------------------------

    \questionpart

    Definiamo le capacità dei tagli:
    \[C_{U_1} = C_1 + C_2 = 5\]
    \[C_{U_2} = C_2 + C_3 + C_4 = 7\]
    \[C_{U_3} = C_1 + C_5 = 5\]
    \[C_{U_4} = C_4 + C_5 = 5\]
    Affinché venga aumentato il throughput, per il teorema del \textit{max flow - min cut}, è necessario aumentare la capacità del minimo taglio \(C_{\min}\).
    Come si può notare, si hanno tre minimi tagli distinti e non c'è un arco comune a tutti, quindi non è possibile aumentare il throughput del grafo con una sola unità di capacità. Per cui \(\tau = C_{\min } = 5\)

    %---------------------------------------------------

    \questionpart
    Le allocazioni ottimali per 2 unità di capacità sono le seguenti:

    \begin{description}
        \item[1.] \(+ 1\) su \(e_1\), \(+ 1\) su \(e_4\) \(\implies \tau = C_{\min} = 6\);
        \item[2.] \(+ 1\) su \(e_1\), \(+ 1\) su \(e_5\) \(\implies \tau = C_{\min} = 6\);
        \item[3.] \(+ 1\) su \(e_2\), \(+ 1\) su \(e_5\) \(\implies \tau = C_{\min} = 6\); 
    \end{description}

    Per ricavare queste allocazioni si è agito in modo tale da aumentare la capacità degli archi che concorrono ai tagli di capacità minima. L'opzione in cui vengono allocate 2 unità di capacità ad un solo arco non migliora \(\tau\) dal momento che non esiste un arco comune a tutti e tre i minimi tagli.

    %--------------------------------------------------
    
    \questionpart
    Le allocazioni ottimali per 4 unità di capacità sono le seguenti:

    \begin{description}
        \item[1.] \(+ 1\) su \(e_1\), \(+ 1\) su \(e_2\), \(+ 1\) su \(e_4\), \(+ 1\) su \(e_5\) \(\implies \tau = C_{\min} = 7\);
        \item[2.] \(+ 2\) su \(e_1\), \(+ 2\) su \(e_4\) \(\implies \tau = C_{\min} = 7\);
        \item[3.] \(+ 2\) su \(e_2\), \(+ 2\) su \(e_5\) \(\implies \tau = C_{\min} = 7\);
        \item[4.] \(+ 2\) su \(e_1\), \(+ 2\) su \(e_5\) \(\implies \tau = C_{\min} = 7\);  
    \end{description}

    Tutte e 4 queste allocazioni producono una somma delle capacità dei tagli \(C_{\text{tot}} = 30\).
    \end{alphaparts}