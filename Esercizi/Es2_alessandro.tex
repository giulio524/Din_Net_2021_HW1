\begin{alphaparts}

    \questionpart

    Per il seguente grafo si possono identificare due path distinti tra \(n_1\) e \(n_6\):
    \[p_1 = \{n_1, n_2, n_3, n_6\},\]
    \[p_2 = \{n_1, n_4, n_5, n_6\}.\]

    Definendo con \(z_i\) il flusso sull'i-esimo path andiamo a calcolare il costo totale del sistema:
    \[C = z_1(\underbrace{3z_1+ z_1+ 1+ z_1+ 1}_{ \text{flusso su }p_1})+ z_2(\underbrace{z_2+ 1 +z_2+ 1 + 3z_2}_{ \text{flusso su} p_2}).\]
    Poiché il throughput del grafo è pari a 2, vale:
    \[z_1+ z_2 = 2.\]
    Pertanto, per sostituzione si ottiene:
    \[C = 10z_1^2 - 20z_1 + 24. \]
    \[C' = 0 \implies z_1 = 1,\]
    quindi
    \[z^* = \arg\min C(z) = \begin{pmatrix}
        1\\
        1
    \end{pmatrix}.  \]

    Poiché vale:
    \[ f^* = A^{n_1,n_6}z^*,\]
    \[ f^* = \begin{pmatrix}
        1 & 0 \\
        1 & 0 \\
        1 & 0 \\
        0 & 1 \\
        0 & 1 \\
        0 & 1 
    \end{pmatrix} \begin{pmatrix} 1 \\ 1 \end{pmatrix}
     = \begin{pmatrix}
        1 \\ 1 \\ 1 \\ 1 \\ 1 \\ 1
    \end{pmatrix}. \]

% ------------------------------------------------------------------------------

    \questionpart
    
    Per calcolare l'equilibro di Wardrop se ne esplicitano le condizioni necessarie:

    \[z_1 > 0 \implies \begin{cases} 5z_1 + 2 \leq 5z_2 + 2 \\ 5z_1 + 2 \leq 10- 5z_1 + 2 \end{cases} \xRightarrow[]{z_1+ z_2= 2} z_1 \leq 1,\]
    \[ z_2 > 0 \implies 5z_1 + 2 \geq 5z_2 + 2 \xRightarrow[]{z_1+ z_2= 2} z_1 \geq 1, \]

    quindi \(z_1 = 1.\) Se ne conclude che:
    \[ z^{(0)} = \begin{pmatrix}
        1 \\ 1
    \end{pmatrix}.\]

    Poiché \(z^{(0)} = z^*\) risulta immediato che:
    \[\text{PoA} = 1.\]

    \questionpart
   Inseriamo il seguente link all'interno del grafo:
        \ctikzfig{grafo_es2_b}
    I possibili path all'interno del grafo risultano quindi essere:

    \[p_1 = \{n_1, n_2, n_3, n_6\},\]
    \[p_2 = \{n_1, n_4, n_5, n_6\},\]
    \[p_3 = \{n_1, n_2, n_5, n_6\},\]

    Posto
    \[d_1 = 5\,z_1+3\,z_3 +2,\]
    \[d_2 = 5\,z_2+3\,z_3 +2,\]
    \[d_3 = 7\,z_3+3\,z_2 +3\,z_1,\]
     per calcolare l'equilibro di Wardrop si esplicitano le seguenti condizioni necessarie:
\begin{itemize}
    \item\[z_1 > 0 \implies \begin{cases} d_1 \le d_2, \\ d_1 \le d_3. \end{cases} \]
    La prima equazione può essere riscritta come
    \[d_1 \le d_2 \Leftrightarrow 5\,z_1+3\,z_3 +2 \le 5\,z_2+3\,z_3 +2 \Leftrightarrow z_1 \le z_2, \]
    mentre la seconda diviene
    \[d_1 \le d_3 \Leftrightarrow 5\,z_1+3\,z_3 +2 \le 7\,z_3+3\,z_2 +3\,z_1 \Leftrightarrow  
    2\,z_1 -3\,z_2 -4\,z_3 +2 \le 0.\]
    Pertanto
    \[2\,z_1 -3\,z_2 -4\,z_3 +2 \le 0\xRightarrow[]{z_1+ z_2 + z_3= 2} 6\,z_1 + z_2 -6 \le 0\xRightarrow[]{z_1 \le z_2} z_1 \le \frac{6}{7}.\]
    
    \item \[z_2 > 0 \implies \begin{cases} d_2 \le d_1, \\ d_2 \le d_3. \end{cases} \]
     La prima equazione può essere riscritta come
    \[d_1 \ge d_2 \Leftrightarrow z_1 \ge z_2 \Rightarrow z_1 = z_2, \]
    mentre la seconda diviene
    \[d_2 \le d_3 \Leftrightarrow 5\,z_2+3\,z_3 +2 \le 7\,z_3+3\,z_2 +3\,z_1 \Leftrightarrow  
    3\,z_1 -2\,z_2 +4\,z_3 -2 \ge 0.\]
    Pertanto
    \[3\,z_1 -2\,z_2 +4\,z_3 -2 \ge 0\xRightarrow[]{z_1+ z_2 + z_3= 2} z_1 + 6\,z_2 -6 \le 0\xRightarrow[]{z_1 \ge z_2} z_2 \le \frac{6}{7}.\]
    
    \item \[z_3 > 0 \implies \begin{cases} d_3 \le d_1, \\ d_2 \ge d_3. \end{cases} \]
     La prima equazione può essere riscritta come
    \[d_3 \le d_1 \Leftrightarrow 5\,z_1+3\,z_3 +2 \ge 7\,z_3+3\,z_2 +3\,z_1 \Leftrightarrow 
    2\,z_1-3\,z_2-4\,z_3+2\ge 0 \]
    mentre la seconda diviene
    \[d_2 \le d_3 \Leftrightarrow 3\,z_1 -2\,z_2 +4\,z_3 -2 \le 0.\]
    Pertanto
    \[ \begin{cases} 2\,z_1-3\,z_2-4\,z_3+2\ge 0, \\ 3\,z_1 -2\,z_2 +4\,z_3 -2 \le 0, \end{cases} \xRightarrow[]{z_1+ z_2 + z_3= 2} \begin{cases} 6\,z_1+z_2-6\ge 0, \\ z_1+6\,z_2-6\ge 0. \end{cases} \]
    Poiché nei punti precedenti si è ottenuto che \(z_1 = z_2\), allora
    \[ \begin{cases} z_1 \ge \frac{6}{7}, \\ z_2 \ge \frac{6}{7}. \end{cases}\]
    \end{itemize}
    Di conseguenza
    \[z_1 = z_2 = \frac{6}{7} \xRightarrow[]{z_1+ z_2 + z_3= 2} z_3 = \frac{2}{7},\]
    e quindi posto 
    \[ z^{(0)} = \begin{pmatrix}
       6/7  \\ 6/7 \\ 2/7
    \end{pmatrix},\]
    il vettore di flusso 
    \[ f^{(0)} = A^{n_1,n_6}z^{(0)},\]
    \[ f^{(0)} = \begin{pmatrix}
        1 & 0 & 1\\
        1 & 0 & 0\\
        1 & 0 & 0\\
        0 & 1 & 0\\
        0 & 1 & 1\\
        0 & 0 & 1
    \end{pmatrix} \begin{pmatrix} 6/7  \\ 6/7 \\ 2/7 \end{pmatrix}
     = \begin{pmatrix}
        8/7 \\ 6/7 \\ 6/7 \\ 6/7 \\ 6/7 \\ 8/7 \\ 2/7
    \end{pmatrix}, \]
    è un equilibrio di Wardrop.
    Il costo totale corrispondente è
    \[3\,\frac{64}{49} + \left(\frac{36}{49} + \frac{42}{49} \right) + \left( \frac{36}{49} + \frac{42}{49} \right) + \left( \frac{36}{49} + \frac{42}{49} \right) + \]
    \[ + \left( \frac{36}{49} + \frac{42}{49} \right) + 3\,\frac{64}{49} + \frac{4}{49} = \frac{100}{7} \simeq 14.3. \]
    Nel caso precedente (grafo senza il link \(e_7\)) poiché
     \[ f^{(0)} = \begin{pmatrix}
        1 \\ 1 \\ 1 \\ 1 \\ 1 \\ 1
    \end{pmatrix},\]
    il costo corrispondente era pari a 
    \[ 3 + 2 + 2 + 2 + 2 + 3 = 14, \]
    pertanto, aggiungendo il link \(e_7\) si è aumentato il costo totale e quindi si è verificato il paradosso di Braess.
    
    L'ottimo di sistema del nuovo grafo è la soluzione del seguente problema di ottimizzazione
    \[ \text{minimize} \quad 5\,z_1^2 + 3\,z_3\,z_1 + 2\,z_1 + 5\,z_2^2 + 3\,z_3\,z_2 + 2\,z_2 + 7\,z_3^2 + 3\,z_2\,z_3 + 3\,z_3\,z_1,\]
    \[ \text{subject to} \quad z_1 + z_2 + z_3 = 2,\]
    il quale ha come soluzione
     \[ z^{*} = \begin{pmatrix}
        1 \\ 1 \\ 0
    \end{pmatrix}.\]
    Di conseguenza, il prezzo dell'anarchia del nuovo grafo è
    \[ \frac{\frac{100}{7}}{14} = \frac{100}{98} \simeq 1.02.\]
    
% ------------------------------------------------------------------------------

    \questionpart
    Dal punto precedente 
       \[ f^{*} = \begin{pmatrix}
        1 & 0 & 1\\
        1 & 0 & 0\\
        1 & 0 & 0\\
        0 & 1 & 0\\
        0 & 1 & 1\\
        0 & 0 & 1
    \end{pmatrix} \begin{pmatrix} 1  \\ 1 \\ 0 \end{pmatrix}
     = \begin{pmatrix}
        1 \\ 1 \\ 1 \\ 1 \\ 1 \\ 1 \\ 0
    \end{pmatrix}, \]
    poiché è noto che 
    \begin{align} &\omega_{e}^{*} = f_{e}^{*}\,d_{e} \left(f_{e}^{*}\right),&& e\in\mathcal{E}, \end{align}
    si ricava che 
    \[ \omega_1 = 3, \,
    \omega_2 = 1, \,
    \omega_3 = 1, \,
    \omega_4 = 1, \,
    \omega_5 = 1, \,
    \omega_6 = 3, \,
    \omega_7 = 0, \,
    \]
    per cui 
    \[ \omega^{*} = \begin{pmatrix}
        3 \\ 1 \\ 1 \\ 1 \\ 1 \\ 3 \\ 0
    \end{pmatrix}\]
    è la distribuzione ottimale dei pedaggi.
    Al fine di calcolare una distribuzione ottimale dei pedaggi a supporto pieno si consideri il pedaggio \(\omega^{*} + a\mathbbm{1}\). Intuitivamente, quello che si sta facendo è aggiungere lo stesso pedaggio costante, pari ad \(a>0\), ad ogni link. Poiché tutti i path hanno la stessa lunghezza pari a $3$, questo fatto non dovrebbe influire sulla scelta della strada da parte degli utenti. Sarebbe come se gli utenti dovessero pagare un pedaggio pari a $3a$ per accedere alla rete, indipendentemente dal path scelto.

A questo punto osserviamo che, preso \(\omega^*\in\mathbb{R}_{+}^{\mathcal{E}} : f^{(\omega^*)} = f^{*}\), allora
\[ f^{(\omega^{*} + a\mathbbm{1})} = f^{(\omega^*)} = f^{*}. \]
Sappiamo, infatti, che $f^{(\omega^*)}$ è un equilibrio di Wardrop per un pedaggio $\omega^*$ quindi
\[z_p>0\implies\]
\[\implies \sum_{e\in\mathcal{E}} A_{ep}^{(n1,n6)} \left(d_{e}\left(f_{e}^{(\omega^*)}\right)+\omega_{e}^{*}\right) \le \sum_{e\in\mathcal{E}} A_{eq}^{(n1,n6)} \left(d_{e}\left(f_{e}^{(\omega^*)}\right)+\omega_{e}^{*}\right), \forall q \in \Gamma_{n1,n6}. \]
D'altra parte anche  $f^{(\omega^{*} + a\mathbbm{1})}$ è un equilibrio di Wardrop per un pedaggio $\omega^{*} + a\mathbbm{1}$, per cui \( \forall q \in \Gamma_{n1,n6},\)
\[z_p>0\implies\]
\[\sum_{e\in\mathcal{E}} A_{ep}^{(n1,n6)} \left(d_{e}\left(f_{e}^{(\omega^{*} + a\mathbbm{1})}\right)+\omega_{e}^{*} + a \right) \le \sum_{e\in\mathcal{E}} A_{eq}^{(n1,n6)} \left(d_{e}\left(f_{e}^{(\omega^{*} + a\mathbbm{1})}\right)+\omega_{e}^{*} + a \right). \]
Tuttavia, quest'ultima condizione può essere riscritta come
\begin{align}
&\sum_{e\in\mathcal{E}} A_{ep}^{(n1,n6)} \left(d_{e}\left(f_{e}^{(\omega^{*} + a\mathbbm{1})}\right)+\omega_{e}^{*} \right)+ a\sum_{e\in\mathcal{E}} A_{ep}^{(n1,n6)} \nonumber\\ 
&\le \sum_{e\in\mathcal{E}} A_{eq}^{(n1,n6)} \left(d_{e}\left(f_{e}^{(\omega^{*} + a\mathbbm{1})}\right)+\omega_{e}^{*}\right) + a\sum_{e\in\mathcal{E}} A_{eq}^{(n1,n6)} , \hspace{0.5cm} \forall q \in \Gamma_{n1,n6},\nonumber\\
&\sum_{e\in\mathcal{E}} A_{ep}^{(n1,n6)} \left(d_{e}\left(f_{e}^{(\omega^{*} + a\mathbbm{1})}\right)+\omega_{e}^{*} \right) + 3a \le \sum_{e\in\mathcal{E}} A_{eq}^{(n1,n6)} \left(d_{e}\left(f_{e}^{(\omega^{*} + a\mathbbm{1})}\right)+\omega_{e}^{*}\right) +3a , \hspace{0.5cm} \forall q \in \Gamma_{n1,n6},\nonumber\\
&\sum_{e\in\mathcal{E}} A_{ep}^{(n1,n6)} \left(d_{e}\left(f_{e}^{(\omega^{*} + a\mathbbm{1})}\right)+\omega_{e}^{*} \right) \le \sum_{e\in\mathcal{E}} A_{eq}^{(n1,n6)} \left(d_{e}\left(f_{e}^{(\omega^{*} + a\mathbbm{1})}\right)+\omega_{e}^{*}\right) , \hspace{0.5cm} \forall q \in \Gamma_{n1,n6}.\nonumber
\end{align}
Poiché per il grafo in esame le funzioni di ritardo sugli edge $d_{e}$, definite in precedenza, sono tutte funzioni crescenti questo ci dice che l'equilibrio di Wardrop per un dato $\omega\in\mathbb{R}_{+}$ è unico, e quindi se vale la disuguaglianza precedente si ha che
\[f^{(\omega^{*} + a\mathbbm{1})} = f^{(\omega^*)}.\]
Pertanto, il vettore di pedaggi 
\begin{align} 
\omega^{*}_{fs} &=  \omega^{*} + a\mathbbm{1}
 			= \begin{pmatrix}
        				4 \\ 2 \\ 2 \\ 2 \\ 2 \\ 4 \\ 1
    			     \end{pmatrix},&& a>0\nonumber
\end{align}
è un vettore di pedaggi con supporto pieno il cui vettore dei flussi ottimale corrispondente riduce il PoA a $1$.
 
 Al fine di trovare un vettore di pedaggi di supporto minimo si osserva che \(\text{PoA} = 1\) se 
 \[ f^{(\omega)} = f^{*}\]
 ossia
 \[z^{(\omega)} = z^{*}.\]
Ricordando che
 \[z^{*} = \begin{pmatrix}
        				1 \\ 1 \\ 0
    			     \end{pmatrix} \]
si nota che, nel rispetto dell'ottimo di sistema, nessun utente sceglie di percorrere il path $p_3$. Pertanto, un $\omega_{ms}^{*}$ di supporto minimo tale per cui gli utenti non hanno alcun motivo di scegliere il path $3$ produrrebbe quindi \(z^{(\omega_{ms}^{*})} = z^{*}\). Si sceglie quindi
\begin{align}&\omega_{ms}^{*} =  \begin{pmatrix}
        				0 \\ 0 \\  0 \\ 0 \\ 0 \\ 0 \\ k
    			     \end{pmatrix},&&k>0.\nonumber \end{align}
Per determinare il $k$ più opportuno si ragiona nel modo seguente:
\[ z_{3}^{(\omega_{ms}^{*})} > 0 \implies \begin{cases} 3\left(z_1 + z_3\right) + z_3 + k +3\left(z_2 + z_3\right) \le 3\left(z_1 + z_3 \right) + z_1 + 1 + z_1 + 1, \\ 3\left(z_1 + z_3\right) + z_3 + k +3\left(z_2 + z_3\right) \le  z_2 + 1 + z_2 + 1 + 3\left(z_2 + z_3 \right), \end{cases} \]
\[ \Leftrightarrow \begin{cases} z_3 \le \frac{2\,z_1-3\,z_2+2-k}{4} \\ z_3 \le \frac{2\,z_2-3\,z_1+2-k}{4} \end{cases} \xRightarrow[]{z_1\le 2 \,\land\, z_2\le 2} \begin{cases} z_3 \le \frac{6-3\,z_2-k}{4} \\ z_3 \le \frac{6-3\,z_1-k}{4} \end{cases}\xRightarrow[]{z_1\ge 0 \,\land\, z_2 \ge 0}\begin{cases} z_3 \le \frac{6-k}{4} \\ z_3 \le \frac{6-k}{4} \end{cases}.\]
Pertanto, se si sceglie $k=6$ allora 
\[z_{3}^{(\omega_{ms}^{*})} \le 0 \implies  z_{3}^{(\omega_{ms}^{*})} = 0,\]
e quindi
\[z^{(\omega_{ms}^{*})} = z^{*}= \begin{pmatrix}
        				1 \\ 1 \\ 0
    			     \end{pmatrix}.\]
			     
% ------------------------------------------------------------------------------

    \questionpart
In questo caso il nuovo grafo presenta dei ritardi ancora minori sui path $p_1$ e $p_2$, in quanto il ritardo è diminuito di due unità su entrambi i path. L'incentivo a prendere il path $p_3$ diminuisce, pertanto è verosimile che l'ottimo di sistema del nuovo grafo sia ancora 
\[ z^{*} = \begin{pmatrix}
        1 \\ 1 \\ 0
    \end{pmatrix},\]
in quanto, se prima non c'era motivo di prendere il path $p_3$, ora è ancora meno conveniente.

Quindi una distribuzione di pedaggi tale che \(\text{PoA} = 1\) è quella in cui il pedaggio sul link $e_7$ è talmente alto che non è conveniente percorrerlo anche se vuoto. Quindi, analogamente al punto precedente si sceglie 
\begin{align}&\omega =  \begin{pmatrix}
        				0 \\ 0 \\  0 \\ 0 \\ 0 \\ 0 \\ a
    			     \end{pmatrix},&&a>0.\nonumber \end{align}
Per determinare il $k$ più opportuno si ragiona nel modo seguente:
\[ z_{3}^{(\omega)} > 0 \implies \begin{cases} 3\left(z_1 + z_3\right) + z_3 + a +3\left(z_2 + z_3\right) \le 3\left(z_1 + z_3 \right) + 2\,z_1, \\ 3\left(z_1 + z_3\right) + z_3 + a +3\left(z_2 + z_3\right) \le  2\,z_2 + 3\left(z_2 + z_3 \right), \end{cases} \]
\[ \Leftrightarrow \begin{cases} z_3 \le \frac{2\,z_1-3\,z_2-a}{4} \\ z_3 \le \frac{2\,z_2-3\,z_1-a}{4} \end{cases} \xRightarrow[]{0\le z_e \le 2, \, e\in\{1,2\}}\begin{cases} z_3 \le \frac{4-a}{4} \\ z_3 \le \frac{4-a}{4} \end{cases}.\]
Pertanto, se si sceglie $a=4$ allora 
\[z_{3}^{(\omega)} \le 0 \implies  z_{3}^{(\omega)} = 0,\]
per cui la distribuzione di pedaggi 
\[\omega =  \begin{pmatrix}
        				0 \\ 0 \\  0 \\ 0 \\ 0 \\ 0 \\ a
    			     \end{pmatrix},\]
è ottimale, infatti
\[ z^{\omega} = \begin{pmatrix}
        1 \\ 1 \\ 0
    \end{pmatrix}.\]


\end{alphaparts}